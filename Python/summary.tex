%%%%%%%%%%%%%%%%%%%%%%%%%%%%%%%%%%%%%%%%%
% Stylish Article
% LaTeX Template
% Version 2.0 (13/4/14)
%
% This template has been downloaded from:
% http://www.LaTeXTemplates.com
%
% Original author:
% Mathias Legrand (legrand.mathias@gmail.com)
%
% License:
% CC BY-NC-SA 3.0 (http://creativecommons.org/licenses/by-nc-sa/3.0/)
%
%%%%%%%%%%%%%%%%%%%%%%%%%%%%%%%%%%%%%%%%%

%----------------------------------------------------------------------------------------
%	PACKAGES AND OTHER DOCUMENT CONFIGURATIONS
%----------------------------------------------------------------------------------------

\documentclass[fleqn,10pt]{SelfArx} % Document font size and equations flushed left

\usepackage{lipsum} % Required to insert dummy text. To be removed otherwise

%----------------------------------------------------------------------------------------
%	COLUMNS
%----------------------------------------------------------------------------------------

\setlength{\columnsep}{0.55cm} % Distance between the two columns of text
\setlength{\fboxrule}{0.75pt} % Width of the border around the abstract

%----------------------------------------------------------------------------------------
%	COLORS
%----------------------------------------------------------------------------------------

\definecolor{color1}{RGB}{0,0,90} % Color of the article title and sections
\definecolor{color2}{RGB}{0,20,20} % Color of the boxes behind the abstract and headings

%----------------------------------------------------------------------------------------
%	HYPERLINKS
%----------------------------------------------------------------------------------------

\usepackage{hyperref} % Required for hyperlinks
\hypersetup{hidelinks,colorlinks,breaklinks=true,urlcolor=color2,citecolor=color1,linkcolor=color1,bookmarksopen=false,pdftitle={Title},pdfauthor={Author}}

\usepackage{tikz}
\usetikzlibrary{shapes,arrows}
\usepackage{listings}
\usepackage{xcolor}
\lstset{ 
	language=Python, % choose the language of the code
	basicstyle=\scriptsize,
	keywordstyle=\color{black}\bfseries, % style for keywords
	numbers=none, % where to put the line-numbers
	numberstyle=\tiny, % the size of the fonts that are used for the line-numbers     
	backgroundcolor=\color{white},
	showspaces=false, % show spaces adding particular underscores
	showstringspaces=false, % underline spaces within strings
	showtabs=false, % show tabs within strings adding particular underscores
	frame=single, % adds a frame around the code
	tabsize=2, % sets default tabsize to 2 spaces
	rulesepcolor=\color{gray},
	rulecolor=\color{black},
	captionpos=b, % sets the caption-position to bottom
	breaklines=true, % sets automatic line breaking
	breakatwhitespace=false, 
}



%----------------------------------------------------------------------------------------
%	ARTICLE INFORMATION
%----------------------------------------------------------------------------------------


\PaperTitle{Python 2.7} % Article title

%----------------------------------------------------------------------------------------
%	ABSTRACT
%----------------------------------------------------------------------------------------

\Abstract{Python is a scripting language.}

%----------------------------------------------------------------------------------------

\begin{document}
	
	\flushbottom % Makes all text pages the same height
	
	\maketitle % Print the title and abstract box
	
	\tableofcontents % Print the contents section
	
	\thispagestyle{empty} % Removes page numbering from the first page
	
	%----------------------------------------------------------------------------------------
	%	ARTICLE CONTENTS
	%----------------------------------------------------------------------------------------
	
	\section{Introduction} 
	
	\subsection[Compile and Interpret]{Compile and Interpret\footnote{\url{http://stackoverflow.com/questions/6889747/is-python-interpreted-or-compiled-or-both}} \footnote{\url{https://docs.python.org/2/library/dis.html}}}
	
	\tikzstyle{block} = [draw, fill=blue!20, rectangle, 
	minimum height=2em, minimum width=3em]
	
	\begin{tikzpicture}[auto, node distance=1cm,>=latex']
	\node [block] (Program) {.py file};
	\node [block, right of=Program, node distance=2.5cm] (Compiler) {Compiler};
	\node [block, right of=Compiler, node distance=3.5cm] (Int) {Interpreter};
	
	\draw [->] (Program) -- (Compiler);
	\draw [->] (Compiler) -- node[name=bytecode] {bytecode} (Int);
	
	\end{tikzpicture}
	
	When we run a python script, first it's compiled. Not to machine code but to bytecode. This compilation make .pyc files. Then this code is interpreted in a VM. There are a lot interpreeters of python: CPython, Jython or IronPython.
			
	%------------------------------------------------
	
	\section[Variables and Definitions]{Variables and Definitions\footnote{\url{https://docs.python.org/2/tutorial/index.html}}}  
	
	\subsection{Variables}
	There is not types. The definition is just the name and the value.

	\begin{lstlisting}
n = 24
s = 'Hola'
a = ['a', 'b', 'c']
>>> range(5, 10)
[5, 6, 7, 8, 9]
	\end{lstlisting}
In while statement, continue statement continues with the next iteration.  

	\subsection{Function}
	
	\begin{lstlisting}[language=Python, basicstyle=\scriptsize]
>>> def fib(n = 1):    # write Fibonacci series up to n
...     """Print a Fibonacci series up to n."""
...     a, b = 0, 1
...     while a < n:
...         print a,
...         a, b = b, a+b
...
>>> fib(2000)
0 1 1 2 3 5 8 13 21 34 55 89 144 233 377 610 987 1597
>>> fib
<function fib at 10042ed0>
>>> f = fib
>>> f(100)
0 1 1 2 3 5 8 13 21 34 55 89
>>> f() # NEVER modify the default parameter (NEVER DO n=...) 
1
	\end{lstlisting}
	\subsubsection{Lambda functions}
	\begin{lstlisting}[language=Python, basicstyle=\scriptsize]
>>> def make_incrementor(n):
...     return lambda x: x + n
...
>>> f = make_incrementor(42)
>>> f(0)
42
>>> f(1)
43
	\end{lstlisting}

	\subsection[Scope]{Scope\footnote{\url{http://stackoverflow.com/questions/291978/short-description-of-python-scoping-rules}}}
	
	\textbf{LEGB Rule:} define the order of search variable.
	\begin{itemize}
		\item L. Local. (Names assigned in any way within a function (def or lambda)), and not declared global in that function.
		
		\item E. Enclosing function locals. (Name in the local scope of any and all enclosing functions (def or lambda), form inner to outer.
		
		\item G. Global (module). Names assigned at the top-level of a module file, or declared global in a def within the file.
		
		\item B. Built-in (Python). Names preassigned in the built-in names module : open,range,SyntaxError,...
	\end{itemize}
	
	The for loop does not have it's own namespace. It would look in the LEGB order.
	
	So first search local variables, then enclosing function locals, Global variable and finally built-in functions.

	\subsection{Modules}
	If we have the definition of fib function in fibo.py file, we can use it like a module definiing it like this.
	\begin{lstlisting}[language=Python, basicstyle=\scriptsize]
>>> import fibo
>>> fibo.fib(1000)
1 1 2 3 5 8 13 21 34 55 89 144 233 377 610 987
>>> from fibo import fib
>>> fib(500)
1 1 2 3 5 8 13 21 34 55 89 144 233 377
	\end{lstlisting}
	
	\subsection{Classes}

	\begin{lstlisting}[language=Python, basicstyle=\scriptsize]
class Dog:
	kind = 'canine'         # class variable shared 
			# by all instances

	def __init__(self, name):
		self.name = name    # instance variable 
			# unique to each instance

>>> d = Dog('Fido')
>>> e = Dog('Buddy')
>>> d.kind                  # shared by all dogs
'canine'
>>> e.kind                  # shared by all dogs
'canine'
>>> d.name                  # unique to d
'Fido'
>>> e.name                  # unique to e
'Buddy'
	\end{lstlisting}
	
	\subsubsection{Inheritance}
	\begin{lstlisting}[language=Python, basicstyle=\scriptsize]
class DerivedClassName(Base1, Base2, Base3):
	<statement-1>
	...
	<statement-N>
	\end{lstlisting}
	
	\subsection{Error and exceptions}

	\begin{lstlisting}[language=Python, basicstyle=\scriptsize]
import sys

try:
	f = open('myfile.txt')
	s = f.readline()
	i = int(s.strip())
except IOError as e:
	print "I/O error({0}): {1}".format(e.errno, e.strerror)
except ValueError:
	print "Could not convert data to an integer."
except:
	print "Unexpected error:", sys.exc_info()[0]
raise
	\end{lstlisting}
	
	\subsubsection{Raising exception}
	The raise statement allows the programmer to force a specified exception to occur. 
	\begin{lstlisting}[language=Python, basicstyle=\scriptsize]
>>> raise NameError('HiThere')
Traceback (most recent call last):
File "<stdin>", line 1, in ?
NameError: HiThere
	\end{lstlisting}
	
	%------------------------------------------------
	
	\section{The Python Standard Library} 
	
	Reference: \url{https://docs.python.org/2/library/}
	
	%------------------------------------------------
	
	\section{Magic methods}

	Reference: \url{http://www.rafekettler.com/magicmethods.html}
	%------------------------------------------------	
	
	\section{Methods}

	\begin{equation}
	\cos^3 \theta =\frac{1}{4}\cos\theta+\frac{3}{4}\cos 3\theta
	\label{eq:refname2}
	\end{equation}
	
	\lipsum[5] % Dummy text
	
	\begin{enumerate}[noitemsep] % [noitemsep] removes whitespace between the items for a compact look
		\item First item in a list
		\item Second item in a list
		\item Third item in a list
	\end{enumerate}
	
	\subsection{Subsection}
	
	\lipsum[6] % Dummy text
	
	\paragraph{Paragraph} \lipsum[7] % Dummy text
	\paragraph{Paragraph} \lipsum[8] % Dummy text
	
	\subsection{Subsection}
	
	\lipsum[9] % Dummy text
		
	%------------------------------------------------
	
	\section{Results and Discussion}
	
	\lipsum[10] % Dummy text
	
	\subsection{Subsection}
	
	\lipsum[11] % Dummy text
	
	\begin{table}[hbt]
		\caption{Table of Grades}
		\centering
		\begin{tabular}{llr}
			\toprule
			\multicolumn{2}{c}{Name} \\
			\cmidrule(r){1-2}
			First name & Last Name & Grade \\
			\midrule
			John & Doe & $7.5$ \\
			Richard & Miles & $2$ \\
			\bottomrule
		\end{tabular}
		\label{tab:label}
	\end{table}
	
	\subsubsection{Subsubsection}
	
	\lipsum[12] % Dummy text
	
	\begin{description}
		\item[Word] Definition
		\item[Concept] Explanation
		\item[Idea] Text
	\end{description}
	
	\subsubsection{Subsubsection}
	
	\lipsum[13] % Dummy text
	
	\begin{itemize}[noitemsep] % [noitemsep] removes whitespace between the items for a compact look
		\item First item in a list
		\item Second item in a list
		\item Third item in a list
	\end{itemize}
	
	\subsubsection{Subsubsection}
	
	\lipsum[14] % Dummy text
	
	\subsection{Subsection}
	
	\lipsum[15-23] % Dummy text
	
\end{document}