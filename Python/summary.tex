%%%%%%%%%%%%%%%%%%%%%%%%%%%%%%%%%%%%%%%%%
% Stylish Article
% LaTeX Template
% Version 2.0 (13/4/14)
%
% This template has been downloaded from:
% http://www.LaTeXTemplates.com
%
% Original author:
% Mathias Legrand (legrand.mathias@gmail.com)
%
% License:
% CC BY-NC-SA 3.0 (http://creativecommons.org/licenses/by-nc-sa/3.0/)
%
%%%%%%%%%%%%%%%%%%%%%%%%%%%%%%%%%%%%%%%%%

%----------------------------------------------------------------------------------------
%	PACKAGES AND OTHER DOCUMENT CONFIGURATIONS
%----------------------------------------------------------------------------------------

\documentclass[fleqn,10pt]{SelfArx} % Document font size and equations flushed left

\usepackage{lipsum} % Required to insert dummy text. To be removed otherwise

%----------------------------------------------------------------------------------------
%	COLUMNS
%----------------------------------------------------------------------------------------

\setlength{\columnsep}{0.55cm} % Distance between the two columns of text
\setlength{\fboxrule}{0.75pt} % Width of the border around the abstract

%----------------------------------------------------------------------------------------
%	COLORS
%----------------------------------------------------------------------------------------

\definecolor{color1}{RGB}{0,0,90} % Color of the article title and sections
\definecolor{color2}{RGB}{0,20,20} % Color of the boxes behind the abstract and headings

%----------------------------------------------------------------------------------------
%	HYPERLINKS
%----------------------------------------------------------------------------------------

\usepackage{hyperref} % Required for hyperlinks
\hypersetup{hidelinks,colorlinks,breaklinks=true,urlcolor=color2,citecolor=color1,linkcolor=color1,bookmarksopen=false,pdftitle={Title},pdfauthor={Author}}

\usepackage{tikz}
\usetikzlibrary{shapes,arrows}
\usepackage{listings}
\usepackage{xcolor}
\lstset{ 
	language=Python, % choose the language of the code
	basicstyle=\scriptsize,
	keywordstyle=\color{black}\bfseries, % style for keywords
	numbers=none, % where to put the line-numbers
	numberstyle=\tiny, % the size of the fonts that are used for the line-numbers     
	backgroundcolor=\color{white},
	showspaces=false, % show spaces adding particular underscores
	showstringspaces=false, % underline spaces within strings
	showtabs=false, % show tabs within strings adding particular underscores
	frame=single, % adds a frame around the code
	tabsize=2, % sets default tabsize to 2 spaces
	rulesepcolor=\color{gray},
	rulecolor=\color{black},
	captionpos=b, % sets the caption-position to bottom
	breaklines=true, % sets automatic line breaking
	breakatwhitespace=false, 
}



%----------------------------------------------------------------------------------------
%	ARTICLE INFORMATION
%----------------------------------------------------------------------------------------


\PaperTitle{Python 2.7} % Article title

%----------------------------------------------------------------------------------------
%	ABSTRACT
%----------------------------------------------------------------------------------------

\Abstract{Python is a scripting language.}

%----------------------------------------------------------------------------------------

\begin{document}
	
	\flushbottom % Makes all text pages the same height
	
	\maketitle % Print the title and abstract box
	
	\tableofcontents % Print the contents section
	
	\thispagestyle{empty} % Removes page numbering from the first page
	
	%----------------------------------------------------------------------------------------
	%	ARTICLE CONTENTS
	%----------------------------------------------------------------------------------------
	
	\section{Introduction} 
	
	\subsection[Compile and Interpret]{Compile and Interpret\footnote{\url{http://stackoverflow.com/questions/6889747/is-python-interpreted-or-compiled-or-both}} \footnote{\url{https://docs.python.org/2/library/dis.html}}}
	
	\tikzstyle{block} = [draw, fill=blue!20, rectangle, 
	minimum height=2em, minimum width=3em]
	
	\begin{tikzpicture}[auto, node distance=1cm,>=latex']
	\node [block] (Program) {.py file};
	\node [block, right of=Program, node distance=2.5cm] (Compiler) {Compiler};
	\node [block, right of=Compiler, node distance=3.5cm] (Int) {Interpreter};
	
	\draw [->] (Program) -- (Compiler);
	\draw [->] (Compiler) -- node[name=bytecode] {bytecode} (Int);
	
	\end{tikzpicture}
	
	When we run a python script, first it's compiled. Not to machine code but to bytecode. This compilation make .pyc files. Then this code is interpreted in a VM. There are a lot interpreeters of python: CPython, Jython or IronPython.
			
	%------------------------------------------------
	
	\section[Variables and Definitions]{Variables and Definitions\footnote{\url{https://docs.python.org/2/tutorial/index.html}}}  
	
	\subsection{Variables}
	There is not types. The definition is just the name and the value.

	\begin{lstlisting}
n = 24
s = 'Hola'
a = ['a', 'b', 'c']
>>> range(5, 10)
[5, 6, 7, 8, 9]
	\end{lstlisting}
In while statement, continue statement continues with the next iteration.  

	\subsection{Function}
	
	\begin{lstlisting}[language=Python, basicstyle=\scriptsize]
>>> def fib(n = 1):    # write Fibonacci series up to n
...     """Print a Fibonacci series up to n."""
...     a, b = 0, 1
...     while a < n:
...         print a,
...         a, b = b, a+b
...
>>> fib(2000)
0 1 1 2 3 5 8 13 21 34 55 89 144 233 377 610 987 1597
>>> fib
<function fib at 10042ed0>
>>> f = fib
>>> f(100)
0 1 1 2 3 5 8 13 21 34 55 89
>>> f() # NEVER modify the default parameter (NEVER DO n=...) 
1
	\end{lstlisting}
	\subsubsection{Lambda functions}
	\begin{lstlisting}[language=Python, basicstyle=\scriptsize]
>>> def make_incrementor(n):
...     return lambda x: x + n
...
>>> f = make_incrementor(42)
>>> f(0)
42
>>> f(1)
43
	\end{lstlisting}

	\subsection[Scope]{Scope\footnote{\url{http://stackoverflow.com/questions/291978/short-description-of-python-scoping-rules}}}
	
	\textbf{LEGB Rule:} define the order of search variable.
	\begin{itemize}
		\item L. Local. (Names assigned in any way within a function (def or lambda)), and not declared global in that function.
		
		\item E. Enclosing function locals. (Name in the local scope of any and all enclosing functions (def or lambda), form inner to outer.
		
		\item G. Global (module). Names assigned at the top-level of a module file, or declared global in a def within the file.
		
		\item B. Built-in (Python). Names preassigned in the built-in names module : open,range,SyntaxError,...
	\end{itemize}
	
	The for loop does not have it's own namespace. It would look in the LEGB order.
	
	So first search local variables, then enclosing function locals, Global variable and finally built-in functions.

	\subsection{Modules}
	If we have the definition of fib function in fibo.py file, we can use it like a module definiing it like this.
	\begin{lstlisting}[language=Python, basicstyle=\scriptsize]
>>> import fibo
>>> fibo.fib(1000)
1 1 2 3 5 8 13 21 34 55 89 144 233 377 610 987
>>> from fibo import fib
>>> fib(500)
1 1 2 3 5 8 13 21 34 55 89 144 233 377
	\end{lstlisting}
	
	\subsection{Classes}

	\begin{lstlisting}[language=Python, basicstyle=\scriptsize]
class Dog:
	kind = 'canine'         # class variable shared 
			# by all instances

	def __init__(self, name):
		self.name = name    # instance variable 
			# unique to each instance

>>> d = Dog('Fido')
>>> e = Dog('Buddy')
>>> d.kind                  # shared by all dogs
'canine'
>>> e.kind                  # shared by all dogs
'canine'
>>> d.name                  # unique to d
'Fido'
>>> e.name                  # unique to e
'Buddy'
	\end{lstlisting}
	
	For private methods and functions, a leading underscore is conventionally added. 
	
	\subsubsection{Multiple Inheritance}
	\begin{lstlisting}[language=Python, basicstyle=\scriptsize]
class DerivedClassName(Base1, Base2, Base3):
	<statement-1>
	...
	<statement-N>
	\end{lstlisting}
	
	\subsection{Error and exceptions}

	\begin{lstlisting}[language=Python, basicstyle=\scriptsize]
import sys

try:
	f = open('myfile.txt')
	s = f.readline()
	i = int(s.strip())
except IOError as e:
	print "I/O error({0}): {1}".format(e.errno, e.strerror)
except ValueError:
	print "Could not convert data to an integer."
except:
	print "Unexpected error:", sys.exc_info()[0]
raise
	\end{lstlisting}
	
	\subsubsection{Raising exception}
	The raise statement allows the programmer to force a specified exception to occur. 
	\begin{lstlisting}[language=Python, basicstyle=\scriptsize]
>>> raise NameError('HiThere')
Traceback (most recent call last):
File "<stdin>", line 1, in ?
NameError: HiThere
	\end{lstlisting}
	
	%------------------------------------------------
	
	\section[The Python Standard Library]{The Python Standard Library\footnote{\url{https://docs.python.org/2/library/}}} 
	
	\subsection[Iterators and Generator]{Iterators and Generator\footnote{Tarek Zidae - Expert Python Programming}}
	
	\begin{lstlisting}[language=Python, basicstyle=\scriptsize]
>>> i = iter('abc')
>>> i.next()
'a'
>>> i.next()
'b'
>>> i.next()
'c'
	\end{lstlisting}
	
	Generator using yield. \textbf{Generators should be considered every time you deal with a function that returns a sequence or works in a loop.}
	
	\begin{lstlisting}
>>> def fibonacci():
... 	a, b = 0, 1
... 	while True:
... 		yield b
... 		a, b = b, a + b
>>> [fib.next() for i in range(10)] 
[3, 5, 8, 13, 21, 34, 55, 89, 144, 233]
	\end{lstlisting}	

	%------------------------------------------------
	
	\section{Magic methods}

	Reference: \url{http://www.rafekettler.com/magicmethods.html}
	%------------------------------------------------	
	
	\section[Design Patterns]{Design Patterns\footnote{Tarek Zidae - Expert Python Programming - Chapter 14}}

	\begin{itemize}
		\item \textbf{Singleton}: restricts instantiation of a class to one object.
		
	\begin{lstlisting}
>>> class Singleton(object):
... 	def __new__(cls, *args, **kw):
... 		if not hasattr(cls, '_instance'):
... 			orig = super(Singleton, cls)
... 			cls._instance = orig.__new__(cls, *args, **kw)
... 		return cls._instance
... 
>>> class MyClass(Singleton):
... 	a = 1
... 
>>> one = MyClass()
>>> two = MyClass()
>>> two.a = 3
>>> one.a
3
	\end{lstlisting}	
		
		\item \textbf{Adapter}: wraps a class or an object A so that it works in a context intended for a class or an object B.
		
	\begin{lstlisting}
>>> from os.path import split, splitext
>>> class DublinCoreAdapter(object):
... 	def __init__(self, filename):
... 		self._filename = filename
... 	def title(self):
... 		return splitext(split(self._filename)[-1])[0]
... 	def creator(self):
... 		return 'Unknown' # we could get it for real
... 	def languages(self):
... 	return ('en',)
>>> class DublinCoreInfo(object):
... 	def summary(self, dc_ob):
... 		print 'Title: %s' % dc_ob.title()
... 		print 'Creator: %s' % dc_ob.creator()
... 		print 'Languages: %s' % \
... 			', '.join(dc_ob.languages())
... 
>>> adapted = DublinCoreAdapter('example.txt')
>>> infos = DublinCoreInfo()
>>> infos.summary(adapted)
Title: example
Creator: Unknown
Languages: en
	\end{lstlisting}
	
	\item \textbf{Facade}: provides a high-level, simpler access to a subsystem. Facade is usually done on existing systems, where a package's frequent usage is synthesized in high-level functions. Usually, no classes are needed to provide such a pattern and simple functions in the \_init\_.py module are sufficient.
	
	
	\item \textbf{Observer}: This is used to notify a list of objects with a state change.
	
	\begin{lstlisting}
>>> class Event(object):
... 	_observers = []
... 	def __init__(self, subject):
... 		self.subject = subject
...
... 	@classmethod
... 	def register(cls, observer):
... 		if observer not in cls._observers:
... 			cls._observers.append(observer)
...
... 	@classmethod
... 	def unregister(cls, observer):
... 		if observer in cls._observers:
... 			self._observers.remove(observer)
...
... 	@classmethod
... 	def notify(cls, subject):
... 		event = cls(subject)
... 		for observer in cls._observers:
... 			observer(event)
>>> class WriteEvent(Event):
... 	def __repr__(self):
... 		return 'WriteEvent'
... 
>>> def log(event):
... print '%s was written' % event.subject
... 
>>> WriteEvent.register(log)
>>> class AnotherObserver(object):
... 	def __call__(self, event): 
... 		print 'Yeah %s told me !' % event
... 
>>> WriteEvent.register(AnotherObserver())
>>> WriteEvent.notify('a given file')
a given file was written
Yeah WriteEvent told me !
	\end{lstlisting}

	\item \textbf{Visitor}: helps in separating algorithms from data structures
	
	\begin{lstlisting}
>>> class Printer(object):
... 	def visit_list(self, ob):
... 		print 'list content :'
... 		print str(ob)
... 	def visit_dict(self, ob):
... 		print 'dict keys: %s' % ','.join(ob.keys())
>>> def visit(visited, visitor):
... 	cls = visited.__class__.__name__
... 	meth = 'visit_%s' % cls
... 	method = getattr(visitor, meth, None)
... 	if meth is None:
... 		meth(visited)
... 
>>> visit([1, 2, 5], Printer())
list content : [1, 2, 5]
>>> visit({'one': 1, 'two': 2, 'three': 3}, Printer())
dict keys: three,two,one
	\end{lstlisting}

	\end{itemize}

	%------------------------------------------------	
	
	
	\section{Advanced topics}
	
	\subsection[Metaclass method]{Metaclass method\footnote{Tarek Zidae - Expert Python Programming - Chapter 3}}
	
	\begin{lstlisting}
>>> def method(self):
... return 1
...
>>> klass = type('MyClass', (object,), {'method': method})
>>> instance = klass()
>>> instance.method()
1
	\end{lstlisting}
		
	%------------------------------------------------

	\section{Unit tests}
	
	\begin{itemize}
		\item PyUnit \url{http://pyunit.sourceforge.net/pyunit.html}
		\item DocTest \url{https://docs.python.org/2/library/doctest.html}
	\end{itemize}

	%------------------------------------------------
	
	\section[Profiling]{Profiling\footnote{\url{https://docs.python.org/2/library/profile.html}}}
	
	\begin{lstlisting}
python -m cProfile [-o output_file] [-s sort_order] myscript.py
	\end{lstlisting}
	
	Memory profiling: use Guppy \footnote{\url{http://guppy-pe.sourceforge.net/}}
	
	%------------------------------------------------
	
	\section{Multiprocessing}
	
	Global Interpreter Lock (GIL) is a mechanism used in computer language interpreters to synchronize the execution of threads so that only one thread can execute at a time. An interpreter which uses GIL will always allow exactly one thread to execute at a time, even if run on a multi-core processor. Some popular interpreters that have GIL are CPython and Ruby MRI.
	
	\begin{lstlisting}
>>> from processing import Process
>>> import os
>>> def work():
... print 'hey i am a process, id: %d' % os.getpid()
>>> ps = []
>>> for i in range(4):
... p = Process(target=work)
... ps.append(p)
... p.start()
... 
hey i am a process, id: 27457
hey i am a process, id: 27458
hey i am a process, id: 27460
	\end{lstlisting}
	
	
	%------------------------------------------------

	\section{Useful Tools}
	
	\begin{itemize}
		\item \textbf{Pylint}, a very flexible source code analyzer
		\item \textbf{CloneDigger}, a duplicate code detection tool
	\end{itemize}
	
	%------------------------------------------------
	
	\section{Books}
	
	\subsection{Read:}
	\begin{itemize}
		\item Tarek Zidae - Expert Python Programming
	\end{itemize}
	
	\subsection{To read:}	
	\begin{itemize}
		\item Code Like a Pythonista: Idiomatic Python \footnote{\url{http://python.net/~goodger/projects/pycon/2007/idiomatic/handout.html#don-t-reinvent-the-wheel}}
		\item Fluent Python - Luciano Ramalho
	\end{itemize}	
	
\end{document}